%!TEX TS-program = xelatex
%!TEX encoding = UTF-8 Unicode
% Awesome CV LaTeX Template for Cover Letter (Modified by qqsugar2020)
%
% This template is based on the original work from:
% https://github.com/posquit0/Awesome-CV
%
% Original Authors:
% Claud D. Park <posquit0.bj@gmail.com>
% Lars Richter <mail@ayeks.de>
%
% Modifications made by qqsugar2020:
% - Chinese font handling improved
% - Customized layout for better support of multilingual content
% - Modification can be seen:https://github.com/QQsugar2000/Awesome-CV_CN/
%
% Template license:
% CC BY-SA 4.0 (https://creativecommons.org/licenses/by-sa/4.0/)
%
% See the LICENSE file for more details.



%-------------------------------------------------------------------------------
% CONFIGURATIONS
%-------------------------------------------------------------------------------
% A4 paper size by default, use 'letterpaper' for US letter
\documentclass[11pt, a4paper]{awesome-cv}

% Configure page margins with geometry
\geometry{left=1.4cm, top=.8cm, right=1.4cm, bottom=1.8cm, footskip=.5cm}

% Specify the location of the included fonts
\fontdir[fonts/]

% Color for highlights
% Awesome Colors: awesome-emerald, awesome-skyblue, awesome-red, awesome-pink, awesome-orange
%                 awesome-nephritis, awesome-concrete, awesome-darknight
\colorlet{awesome}{awesome-red}
% Uncomment if you would like to specify your own color
% \definecolor{awesome}{HTML}{CA63A8}

% Colors for text
% Uncomment if you would like to specify your own color
% \definecolor{darktext}{HTML}{414141}
% \definecolor{text}{HTML}{333333}
% \definecolor{graytext}{HTML}{5D5D5D}
% \definecolor{lighttext}{HTML}{999999}

% Set false if you don't want to highlight section with awesome color
\setbool{acvSectionColorHighlight}{true}

% If you would like to change the social information separator from a pipe (|) to something else
\renewcommand{\acvHeaderSocialSep}{\quad\textbar\quad}


%-------------------------------------------------------------------------------
%	PERSONAL INFORMATION
%	Comment any of the lines below if they are not required
%-------------------------------------------------------------------------------
\photo[circle,noedge,left]{profile}
\name{不愿}{透露}
\position{浙江大学计算机学院{\enskip\cdotp\enskip}计算机技术硕士}
\address{不愿透露, 余杭塘路866号, 西湖区, 杭州市, 310058, 浙江省}

\mobile{(+86) 199-0999-9999}
\email{26666666666@qq.com}
\github{Git: QQsugar2000}
% \gitlab{gitlab-id}
% \stackoverflow{SO-id}{SO-name}
\wechat{WeChat:199099999999}
% \skype{skype-id}
% \reddit{reddit-id}
% \medium{madium-id}
% \googlescholar{googlescholar-id}{name-to-display}
%% \firstname and \lastname will be used
% \googlescholar{googlescholar-id}{}
% \extrainfo{extra informations}

\quote{``本简历内容完全虚构,包含的经验仅限于作者专业和经历,请自行修改"}



%-------------------------------------------------------------------------------
%	LETTER INFORMATION
%	All of the below lines must be filled out
%-------------------------------------------------------------------------------
% The company being applied to
\recipient
  {计算机技术硕士在读}
  {计算机学院\\人工智能所\\研究方向:知识图谱 人工智能}
% The date on the letter, default is the date of compilation
\letterdate{\today}
% The title of the letter
\lettertitle{方向:算法工程师}
% How the letter is opened
\letteropening{尊敬的HR先生/女生}
% How the letter is closed
\letterclosing{真诚的,}
% Any enclosures with the letter
\letterenclosure[附件]{我的个人简历}


%-------------------------------------------------------------------------------
\begin{document}

% Print the header with above personal informations
% Give optional argument to change alignment(C: center, L: left, R: right)
\makecvheader[R]

% Print the footer with 3 arguments(<left>, <center>, <right>)
% Leave any of these blank if they are not needed
\makecvfooter
  {\today}
  {~~~·~~~Cover Letter}
  {}

% Print the title with above letter informations
\makelettertitle

%-------------------------------------------------------------------------------
%	LETTER CONTENT
%-------------------------------------------------------------------------------
\begin{cvletter}

\lettersection{关于我?}
%由于源项目使用英文排版,中文很容易超出页面宽度,所以请使用\\手动换行
你或许需要在这里用一两句话给出自己的基本信息,可能和上面header的个人信息有些重复,但这是必要的。之后,\\你需要指明自己的申请职位,包括可能已经确定的细分部门、实验室等等。之后,可以用一两句话突出自己的关键\\信息或背景(提前指出自己的优势)。如果不知道写什么,作为cover letter的一部分,可以参考正常书信的开头,\\介绍自己、来由等。事实上,国内求职较少使用cover letter,而申请学位、海外求职的情况复杂,所以这里写什么,\\需要具体情况来定。

\lettersection{为什么投递xx?}
%由于源项目使用英文排版,中文很容易超出页面宽度,所以请使用\\手动换行
这里主要是讲述投递对方的动机,(由于源模板是国外的,他们比较看重对双方文化的认同,国内求职不是在意这块,\\可以写一些对对方公司or实验室的了解到的信息,包括主要业务、产品,或是科研方向,自己对产品线或方向的理解等。\\当然了,考虑到认同与否的原因,也应该体现出自己对对方的一些期待(或者是验证公开信息是否真实)之类的。\\从求职的角度来说,一般来说尽量说好话,这也不需要我特别说明。
\\这一段可能还可以更长,但我说的就这些了,所以你会看到我在水行数……
\\水行数…………

\lettersection{我的优势在于?}
%由于源项目使用英文排版,中文很容易超出页面宽度,所以请使用\\手动换行
这里算是cover letter最重要的部分了,考虑到在海外语境下,看完cover letter对方未必还看简历,这里必须/应该\\要把自己的真实优势和背景说明白。从个人申请的经验来说,一般以讲故事/时间线叙述的方式讲清楚自己的一些学历、\\研究背景,然后引入到自己的优势上来。如作为一个半路转码的人,我会写:我本科做了什么什么,发生了什么什么事\\决定转行,但是我的本科的经历,证明了我的学习能力,还给我带来了跨学科背景,这有利于我未来的什么什么项目……\\当然,没有故事也可以不必硬扯

\end{cvletter}


%-------------------------------------------------------------------------------
% Print the signature and enclosures with above letter informations
\makeletterclosing

\end{document}
